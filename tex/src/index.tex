\chapter{On Persons of the Future}

\section{Letter from prison}

Dear stranger,

Tell me your thoughts. Are things actually bad or is it just me?

Everybody says this is normal. This is what’s supposed to happen. Yet it doesn’t feel normal. It feels like slavery. I am a slave. Worse than a slave. A slave both of body and of mind.

Every day, I wake up early. Too early, if you ask me. Starting from 8 am until 2 pm I have work. We get 10-minute breaks every hour, yet we can’t really do anything during those 10 minutes. It’s only allowed to go out to the gated yard. They say it’s for our protection.

This is not a job one can quit. Nor one can change. There is no pay for this job. They say the work itself will pay in the long run.

During work time, we all have to do what they tell us. That’s fine, I can deal with it. The problem is we also have to think what they tell us. This is what’s killing me.

We are micromanaged. Under constant supervision at all times. The supervisor asks us things. This is how they check if we are thinking the things they tell us. They say it’s for our own good.

We’re not allowed to collaborate for the work. It’s the same work for everyone, but everyone has to do it for themselves. They monitor if we collaborate and if we do, then our work is annulled. They say it’s for justice.

Some special times we have to collaborate. Then the work is opposite. One is forced to collaborate, even with workers they don’t like. They say it’s for society.

In comparison to my coworkers, I have it pretty good. I can focus on the thinking material they ask us and I’m pretty good at the work. Also, I’m terrified when they yell at us. So I only do the things they tell us to do. I don’t even dare to think of doing anything else. The cost-benefit analysis was clear: such fear and shame are too much to bear. Admittedly, I’m a coward.

Some workers can’t take the pressure. They either abuse other workers or don’t do the work and the thinking. It usually ends badly. Usually yelling. I don’t mind it so much. Apart from my constantly elevated heartbeat, it also means that we get to have a few seconds off-thinking.

The workers are divided into groups by the year we were born. The thinking and the work are also divided by year. If you’re really bad at the work, then you have to do an entire year’s work again. I had some coworkers that went through that. It’s pretty rare though.

\hyphenation{co-worker}
Some special times, things get extra ugly. Maybe a coworker lashes out and attacks another coworker. Sometimes the supervisors lash out. They can’t take the supervising pressure or just some other random thing in their lives. Who knows? They won’t tell us. They don’t tell us much in general; only what we are allowed to know. Only what’s relevant to the sanctioned work.

If, during work, one needs to go to the bathroom, they have to ask for permission. Sometimes the supervisors decline. Maybe because of important thinking, or just too many people went to the bathroom. There might be a quota.

Walking around your desk or getting up for a stretch is strictly prohibited. These are activities for break time only.

Our lunch break is 25 minutes. We can’t do much. We can’t cook. We have to either buy from the official store inside the enclave or bring food from home. But we can’t microwave it or something.

At 2 pm, the bell rings and the guard opens the gates. This is when we get to go home. Home is much better. Much more relaxed, but we still have to do work. It takes a few hours. After that, we are free. Or at least that’s what they say.

How can one be free if the only thing they can control in their life is a few hours per day? Let alone when they have no money to spend.

Slaves get limited free time per day. It is slaves that do not get paid for work. It is slaves that cannot quit.

Yet slaves are free to at least dream of freedom. They are free to think of it. Only a slave of mind is not. What could be worse than that?

It was the citizens of Airstrip One\footnote{In George Orwell’s \emph{Nineteen Eighty-Four}, Airstrip One is a province of Oceania, the totalitarian superstate where the story takes place.} that had their thoughts controlled and it was with the same techniques: prescribed information to consume, defined language to talk, constant fear of punishment by the authority.

So, how could we ever expect even a remotely balanced person to come out of school?

\section{School’s wrongness}

Experiences and stories, such as the one in the previous chapter, help us understand problems. Yet we frequently need systematic analyses of how something is unsatisfactory in order to progress at solving it.

In the following list, I tried to compile all problems I discovered in my experience being a student in schools of primary, secondary, and tertiary levels.

\emph{Lack of student autonomy}. Every part and limit of the student’s behaviour is defined a priori. Their exact location during school hours is predetermined. Periods of sitting and standing are decided almost by law. Predefined hours of being inside and outside. Predefined and forced single viewpoint: written in the approved book, expressed by the teacher, established by the legislator. Even further: students’ thoughts are mandated to follow and adopt their teacher’s thoughts at all times.

\emph{Failure in diversity management}. Not everyone has the same talents, skills, background, desires. Yet, not only is everything already defined, but it’s also identical for all. Even if people’s characteristics coincide, the timing may not. Still, in the eyes of the educational system, same age means same person.

\emph{Lack of teacher autonomy}. Defining teaching material by the state might have been a good idea in the past, but now, all knowledge is available to everyone—for free—and that changes everything. The role of the teacher is—and has probably always been—nothing more than the role of a repeater. It doesn’t have to be this way and, usually, a small set of creative teachers demonstrates and proves this daily. Forging teacher homogeneity to shield students from bad teachers has resulted in the total restriction of teachers. Inescapably, this has led to a total lack of creativity and teaching becoming a lifeless transfer of information, as opposed to the achievement of quenching one’s thirst for knowledge. Undeniably, one cannot have the power for creation without also having the freedom for destruction.

\emph{Lack of teacher quality assessment.} Measuring is fundamental to improving. Absence of improvement inevitably leads to absence of skill.

\emph{Lack of regard for the teacher's responsibility towards society.} The importance of the work of the teacher is monumentally undervalued. High-quality teachers are intrinsically connected to more educated society members. Extremely important propositions cascade from this; for instance, higher quality representative democracy due to better educated voters.

\emph{Blatant ageism.} Our model is limited to adult teachers teaching young kids. Unfortunately, even kids teaching kids was too bizarre for those who defined the societies we inherit to consider.

\emph{Pointless curricula size.} The quantity of knowledge attempted to be taught to students in primary and secondary education is too immense. Both in terms of a twelve-year educational system and in terms of continuous learning for six or seven hours per day every day.

\emph{Pointless curricula content.} Humanity’s hive mind, the Internet, makes every piece of information available in a few seconds’ time. Many agree: there is no point in learning information; there is point in learning fundamentals.

\emph{Irrelevant methods of teaching.} Lifeless lectures, indifferent presentations, eternal monologues, stressful exams; even digitally assisted learning and/or interactive learning—unfortunately, all of these have turned out to be disappointing and ineffective. True learning happens in one’s mind, alone, with a community of people around for support—not for pouring knowledge into an empty mind-bucket.

\emph{Total disregard for art.} Virtually everything is revolving around STEM. Art, in any form, is completely disfavoured by everyone: teachers, students, parents, lawmakers. Art’s timeless honesty has perpetuates the human era since its inception, yet at some point, it was deemed unworthy for the intellectual foundation of new humans.

\emph{Partial disregard for the humanities.} Along with its aforementioned conceptual subclass, human disciples have fallen out of favour, as being ineffective in building wealth—the core value of the West’s social imaginary.

\emph{Total disregard for physical knowledge.} Disturbingly small amount of school-approved time in exercising, sports, or any other bodily activity.

\emph{Existence of homework.} Many of the hardest working professions allow for leisure time at home; not school though.

\emph{Limited societal framework.} Students are learning from teachers in a restricted facility for an extraneously defined set of hours per day. School has been shaped as a silo, when it could have been shaped as an essential, integrated part of everybody’s—even non-parent non-student adults’—societal life.

\section{Against adult supremacy}

In a few hundred years, when we talk about the past, it’ll be this reflection that will inhabit the social consciousness: how did we ever think it was normal to treat people as slaves just because of their age?

We will look back and disapprove of anyone who forced people to do things they didn’t want to, just because—at that point—these people were alive for less than 18 years. What we think of slave masters now, we’ll think of “adults” then. Because, just like with slaves, there are exactly zero good reasons to force another consciousness to do something they don’t want to do.

\begin{verse}
    By pouring their derision\\
    Upon anything we did\\
    Exposing every weakness\\
    However carefully hidden by the kids\\
    But in the town, it was well known\\
    When they got home at night, their fat and\\
    Psychopathic wives would thrash them\\
    Within inches of their lives

    — \emph{Pink Floyd, The Happiest Days of our Lives, 1979}
\end{verse}

Just like with slave owners, there were some nasty ones, but there were also humane ones. The same parallel difference exists between strict parents and teachers and more lenient and sympathetic ones.

Just like the first people who opposed slavery were considered amusing or funny, most of the readers of this text will smile in response, speculating I must be exaggerating.

But the grim reality of our present era is that there is yet one more, deeply entrenched, hierarchy inside our by-other-metrics progressive Western societies. \emph{Some humans among us are second rate.} They are considered of inferior intellect. Just like slaves were considered the same. They are not equal citizens. They cannot vote, the fundamental—however pretentious—right of democracy has been subtracted from them. Let’s take a moment to grasp this. We consider the era when women could not vote one of inadequate democracy, an era where only half the population was able to take part in defining how we live. Let us realise now that we disregard this class of humans in such a high degree that we do not even consider them as part of the two halves that could \emph{potentially} vote.

In the future, history books will be rewritten and what they will say about women’s suffrage is not that the other half of the population was finally allowed to vote in the 20th century. Instead, history books will say that in the 20th century the second \emph{third} of the population was allowed to vote. It took hundreds of years more for the final third of the population to be considered equal.

It feels like the obsolescence of adulthood is such a radical concept that it will take centuries for it to materialise. Consider the case of slaves. From the beginning of the first human civilisation—across the globe—in one form or another, all peoples had an implementation of the notion of slavery. The worldwide abolition of slavery (which is not complete but at least on a very significant degree done) is a victory of gigantic proportions for humanity. One of similar proportions will be needed for the abolition of adulthood.

I don’t know how. Comparing with the abolitionist movement and the women’s right to vote, we should start talking and writing\footnote{In 2021, John Wall published the book: \emph{Give Children the Vote: On Democratizing Democracy}, ISBN: 978-1350196261.} about it. This can only be regarded as a first step, though. Maybe we’ll never consider new humans equal. Maybe we’ll go in the opposite direction and limit their freedom even more. Neither is the future of humanity predetermined, nor the arrow of progress singular. What will be considered ethical and progressive in a hundred years is at play right now. In other words, if we don’t change our definition of fairness—ourselves—to include new humans’ opinions as a must-have, then the future world will still be fair—just with a different definition of fairness.

Whether we know it or not, it is us who decide what is fair. Not: “we \emph{can} decide what’s fair”. We decide it whether we do it consciously or not.

Let’s own it, then. Let’s think about it hard and let’s define fairness consciously.

\begin{quote}
    Think we must. Let us think in offices; in omnibuses; while we are standing in the crowd watching Coronations and Lord Mayor’s Shows; let us think as we pass the Cenotaph; and in Whitehall; in the gallery of the House of Commons; in the Law Courts; let us think at baptisms and marriages and funerals. Let us never cease from thinking–what is this “civilization” in which we find ourselves? What are these ceremonies and why should we take part in them? What are these professions and why should we make money out of them?

    — \emph{Virginia Woolf, Three Guineas, 1938}
\end{quote}

\section{Advice to new programmers}

In 2010, James Hague, a recovering programmer\footnote{https://prog21.dadgum.com/56.html}, gave some advice\footnote{https://prog21.dadgum.com/80.html} to new programmers.

He said, if you are someone who would write something like this:

\begin{quote}
    Hey everyone! I just learned Erlang / Haskell / Python / [etc.], and now I'm looking for a big project to write in it. If you've got ideas, let me know!
\end{quote}

\dots then, you're doing it wrong. He went on:

\begin{quote}
    There's nothing about solving a problem or overall usefulness or any relevant connection between the application and the interests of the original poster. Would you trust a music notation program developed by a non-musician? A Photoshop clone written by someone who has never used Photoshop professionally?
\end{quote}

What he says is that it’s not worth learning a tool for the sake of it. He suggests to aimless, excited programmers: find a problem first and then figure out how to use tools to solve it. This way you become an expert [to a small, tightly defined domain].

I don’t think this advice to new programmers, who are indeed many times excited and aimless, is good advice. It would have been fine if programming was just a tool. But it’s not. It’s also a craft and craft practicing is meaningful by itself.

What I would suggest, instead, both to my younger self and even to myself now, is the following:

You’ve chosen an extremely fun path! One thing only: don’t expect to find any projects to write the language you learned soon. Write your own projects. \emph{Let them be pointless.} Write an FTP server, write a BitTorrent client, write a PNG parser/library, build a Tetris game, and a Chess engine, and a Space Invaders clone.

All these might be pointless because there are already tons of high quality open source implementations of these programs. They might also be useless; maybe they will have only one user, you.

But this doesn’t really matter—as long as you’re writing code—that’s the fun part of it. Not only that, but while you’re having all this fun, you’re actually learning a ton about building software in the language you’ve chosen. While coding, you may also check out other, existing, open source clones of what you’re making. This way you can see how different problems map to various designs. Through this, you also learn how to read code\footnote{Reading code (both oneself’s and other people’s) is underappreciated when learning to program, yet it is fundamental in building software.}. Whenever you see yet another open source chess engine, you learn to look for specific components and design choices. E.g. how has this programmer solved problem X and how problem Y?

All of this skill practicing on researching, designing interfaces, writing code, reading code, understanding trade-offs, touching a lot of different domains, will mainly be fun but, inadvertently, will also be extremely useful. Just not immediately. Coding, as an intersection of craft and knowledge is not about usefulness anyway. But coding is also about building tools, which is fundamentally about usefulness.

\begin{quote}
    What I’m really concerned about is reaching one person. And that person may be myself for all I know.

    — \emph{Jorge Luis Borges}
\end{quote}

\chapter*{Interlude I:\\Homage to Chocolate}
\addcontentsline{toc}{chapter}{Interlude I: Homage to Chocolate}

Chocolate is a true heavenly wonder of this world.

Every time I see some chocolate, I feel an amazement just by looking, smelling and feeling such a substance. Its texture can be surprisingly described as both rough and soft. Its colour, original and satisfying. Its smell, deep and luscious.

The structure of a chocolate bar is one of its defining characteristics. Rigid and firm, yet one can easily partition it. When looking at the edges of these partitions, a view into the chocolate’s soul surfaces. A view that is authentic, beautiful, and perfect in its imperfectness of the uneven crack.

The way it breaks by hand is the same when it breaks by mouth. Teeth slowly crush the actualised divinity and, with each crack, the sublime material is divided in half. In this very process, inimitable sensations develop. Transcendental flavours fly all around one’s mouth, an explosion of saliva takes place and a hurricane of frenzy cacao intervenes—one nobody wants to stop.

Yet only a handful of seconds later it does. This is one of those moments — how could we ever not appreciate the fact that we have as much of this celestial elixir we want? The limit is in our digestive apparatus, not in the abundance of the world. I now understand why people say we should be more thankful. It’s due to chocolate.

One would think that even the thought of chocolate ending as a resource would upset me. Yet, I feel peace. Even if chocolate completely disappears from the world, I would be serenely jubilant that we had the opportunity to experience it.

Nevertheless, we still have loads of chocolate and every day I can buy a well-sized portion for one unit of currency. Isn’t that even more amazing?

\chapter{All Problems are Interpersonal}

\section{On wasting one’s time}

There is nothing she hates more than “wasting her time”.

Of course, for me, any second spent with her is the farthest thing from waste. But, in my book, time waste is impossible in general. It has been impossible since I realised what the content of life is. Or maybe what it isn’t. It isn’t to produce as much as possible. Yet, the argument of having control over one’s time, I do consider very valid.

What annoys me is waste in the context of capitalistic production. Time not creating, learning, self-improving in any way, she considers “waste”. How annoying. But just like they say\footnote{The following is attributed to William Faulkner: “You don’t love because: you love despite; not for the virtues, but despite the faults.”}: \emph{you love someone despite the faults, not because of the virtues}.

Well, I do love her \emph{despite} her production-focused mindset. But I also do love her \emph{because} of her determination (among other things).

Interestingly, this determination is the other side of the same coin. It is the key skill one needs in order to be productive and capitalistically successful. This confirms that other saying: \emph{someone’s best feature is also their worst}. This feature is her determination, which leads to both being into nonstop production (the “worst feature”) as well as being intelligent and creative (the “best feature”).

\section{The world is based on gifts}

What are we going to do now that AI will take all our jobs?

\subsection{I. Contradictions}

In 1930, John Maynard Keynes predicted\footnote{In his text \emph{Economic Possibilities for our Grandchildren}, written in 1930.} that, in 2030, society would be so productive that we would barely need to work. Presumably, the more technology we create, the more productive we become, the less work we need to do.

Of course, this isn’t what has happened. We do have the most technology ever, yet we work more than ever\footnote{Living in humanity’s most work-intensive era can be a controversial statement. Before dismissing it, let’s consider that a lot of people work today because they want to and not because they have to. The number of people who engage in such a pursuit of achievement, I feel, is more than ever.}. Indeed, we are more productive than ever—and the richest we have ever been—overall—yet inequality is so high that everybody’s experience is that we are poorer than ever.

Similar, from the Chicago school of economics, the theory of supply-side economics\footnote{There is another term: trickle-down economics, popularised more recently, it refers to policies favouring the wealthy bracket of the population with the hope that their wealth will trickle down to the less wealthy.} implies: the more productive businesses are, the more money they will pay their employees. The opposite is what happens in the real world: the more productive a company is (i.e. the more money a company makes), the more powerful it becomes, and thus, the more able it is to get away with paying less.

\subsection{II. Transactions}

Humanity’s economics are—these days—based on a model of transactions. For example, we expect agriculture workers and companies to cultivate food because they will be rewarded. This reward will be in money, which is a universally accepted representation of (a) wealth and (b) appreciation (i.e. receiving money means they do something that matters).

What happens in practice is that people work in a number of different jobs and earn this universally accepted reward, which they can exchange with other materials and/or services that they do not have and desire. For example, they can exchange it with the food that agriculture workers produce.

Now, the problem is that if people develop highly efficient AIs, all jobs will gradually disappear because these AIs will execute more efficiently than any human could ever imagine.

In that case, all these people whose skills have become trivial will become poor. They will have no way of finding rewards, which means they won’t have money to live. Gradually, all money will converge to the AI people (because they create AIs, or because they operate them, or because they own the metal that AIs run on). The non-AI people’s future will be dire.

\subsection{III. Looking at the past}

We might feel there is no other way than the transaction model, but I think that’s only because we are too deep into it to see something else. If we look around, we might find some disturbing absence of transactions, given that market-based rewarding is so important to us.

To start with a simple and almost silly\footnote{Silly ideas can be powerful. It used to be silly to want a computer in your house. To want it in your pocket wasn’t even silly, though, because it was simply unimaginable.} example: we never properly rewarded the people who first found out how we can farm the land and produce a ton of food. This we might consider an advancement made not by one individual but by humanity as a whole. Presumably, though, there were some people or groups of people who figured it out first. Were they fairly rewarded, given the humanity-changing impact and cumulative benefit of their work?

I don’t think they did. More examples: we never properly rewarded the first people who figured out how we can sew clothes and shoes and coats and make tools and houses. We never properly rewarded the people who discovered how electricity works or the people who designed our cities and the streets we walk every day.

This list is truly unending and impossible to complete. People who figured out how the human body works and how to cure diseases, people who built ships and airplanes and an insanely complicated shipping network, people who figured out all technologies required before some other groups were able to put them together into a smartphone or a computer or the internet.

We haven’t even talked about the people who supported these pioneers. People who produced their food and built their houses so they can have time to think of new ideas. And what about the people who supported these inventors—not materially, but—mentally? Whether these were partners or religious entities or anyone or any-thing else.

All these were actually gifts.

People like Archimedes, Leonardo da Vinci, Leonhard Euler, Nikola Tesla and so many more were not motivated by money rewards. They thought and created all they did for reasons that were not transactional. And today all their work is free for us. All their work is a gift.

For this reason, by looking a bit farther into time, I claim that humanity doesn’t operate on a transactional model but on a gift model.

\subsection{IV. Looking at the future}

There exists an example in today’s real-world economy that shows part of my argument. Open source software is inherently based on a non-transactional gift economy since its inception.

It is partly a factor that programmers are weird people who intrinsically enjoy the process of writing code so much that they do it just because. But beyond that, as a programmer, I can truly say that the fact that somebody makes use of code I’ve written makes me proud. The fact that I made something that helped someone—in addition to the fact that a programmer would rather use my code than theirs—is simply cause for celebration. It’s even more than that: motivation to do more of the same.

It’s the same feeling I have when I gift someone something that they really like.

I really think the open source software movement’s model is the future even though some people think it should be the past.

There are many arguments as to how open source is fundamentally broken because it’s free work with no reward. Of course, people working tirelessly two jobs and people getting burned out is without question terrible. But that’s only because there is no space for leisure time in our lives anymore. Everything is becoming extremely efficient. We can only use whatever time we have awake to achieve and produce—or else we become homeless or losers.

There is no time for experiments anymore, no time for failures, no time for gifts.

But I think that’s the solution. To have time only for experiments, only for failures, only for gifts. To realise the advent of a so-called age of leisure.

\subsection{V. Ideology}

The famous quote\footnote{Mark Fisher mentions this quotation in his book \emph{Capitalist Realism}, which he attributes to both Fredric Jameson and Slavoj Žižek. Interestingly, Mark Fisher hanged himself as a result of his struggling with depression. He said: “the pandemic of mental anguish that afflicts our time cannot be properly understood, or healed, if viewed as a private problem suffered by damaged individuals”. Extreme productivity and efficiency, as part of capitalism, not only can lead to destruction through (possibly inevitable) highly advanced/powerful technology but also through the absence of space for doing otherwise (i.e. doing inefficient actions) which leads to depression (\emph{exhaustion}).} that “it’s easier to imagine an end to the world than an end to capitalism” still echoes in my mind every time I encounter economic dead ends and unsustainabilities.

I really do think there must be a lot of alternatives that current ideologies are too strong to allow us to see. Probably because through our current lenses, they appear silly. Or nonsensical, or pointless, or insane, or irresponsible, or utopian, or dystopic.

Whatever the case, it seems we might be forced to choose between the end of capitalism and the end of the world. Maybe we don’t even have to imagine this dilemma because it’s already here. The decision for this choice is humanity’s current challenge. I’m not saying we’ll figure it out—it might even destroy us. That’s why it’s a challenge. It’s a riddle: can we imagine beyond our collective bubble?—and riddles are things that beg for solutions.

\subsection{VI. Alternatives}

There are two perspectives here.

Perspective one is: humanity is a bunch of units who compete with each other over time, money, resources, et alia. Units who have power over AIs are more capable of enslaving units who don’t. If the units with AI powers are peaceful, they can leave the rest to live without AIs. Or maybe units with AI would rather take all resources for themselves and leave nothing to the rest. Whatever the case, it seems likely that a society whose imaginary\footnote{An imaginary of a society is a set of core, underlying values and beliefs which constitute its foundation. They define how people perceive what’s worth doing and what matters in the society. Another definition is “the set of values, institutions, laws, and symbols through which people imagine their social whole”.} is based on competition will engage in some kind of violent resolution for such power imbalance.

Perspective two is: humanity is a bunch of units who cooperate with each other to maximize wellness for everyone, without keeping ledgers as to who did what. No ledgers is the essence of lack of transactions. Maybe they can all coordinate to enable AIs to build whatever all units want without zero-sum systems and power over others.

We are definitely—currently—seeing the world through the first perspective. It seems to me that unless we consciously decide to change to the second kind, we’re highly likely to fight with each other. I’m pretty sure we can change. How can we change? That I will not reveal. Everybody is able to figure it out for themselves. I’m not implying there is a different subjective response for each person. There is one specific solution in my mind that we can all arrive at. I just don’t want to give it away. Maybe I can give a hint by telling a graffiti story.

\section{One thought at a time}

There is a graffiti piece in Finsbury Park that reads “Poor is cool”.

“Poor is definitely not cool”, Charlie responded when she saw this. “Rich and privileged people say that because they’ve never truly felt the pain of being poor.”

When experiencing privilege, I have felt it’s very hard to cede it. What I understand of Charlie’s argument is that if people experience true destitution—and they can go back to their familiar comfort—they will.

Poor people think rich is cool. Rich means comfort, quality of life, pools of options.

“Let’s just get a taxi—for once”, sometimes Charlie suggests. I reject her proposal. “But it’s so much cooler to cycle!” I say.

We cycle everywhere, but sometimes it’s harder than usual. Maybe it’s raining or maybe it’s late or maybe we’re too tired. It requires extra strength to cycle then.

“Every time we cycle, do you wish we were in a car instead?” I ask Charlie. Because I don’t. I try not to, that is. It’s a conscious endeavour to change my dreams.

To be in a car means to drill into the earth, pull out oil, process, distribute, and finally burn. Every single one of these processes is severely detrimental to everyone: the people who do it, the people who are around, the people who use it, and even the non-people beings, plants included—and that’s true at present but also for the far future; CO$_{2}$ remains out by default.

To cycle, on the other hand, means to operate an elegant machine while at the same time becoming stronger and healthier (at least on our cycle-to-commute level). Cycling is antifragile while [fossil fuel] car driving is not only fragile but also unsustainable in an insanely short-term timeframe.

So, this is why cycling is cooler. But why does it matter what’s cooler? Because what we dream while we are free to dream anything is the process which defines our desires. If we don’t control this process, we don’t control our lives.

In other words, if we dream of elite luxuries, then we celebrate people who have material wealth. Inevitably, then, we want to become like them. These role models we admire and aspire to—as we stroll around, when we are thinking without a filter—define who we want to be. If we change these stories, we change our desires.

And if we change our desires, we change the world. So, that’s how we change the world. One thought at a time.

\section{From one to many}

There is merit in being part of a community for a cause. The cause can be anything, from a hobby to a societal issue. Sometimes the creation of the community just springs out of conversations and shared values. Some other times, it starts in the mind of one person. If that’s the case, how do we start from that One person’s thought and end up in a community of shared ownership?

We might unconsciously associate communities with grassroots movements and democratic governance, but this doesn’t have to be the case. Not all communities are democratic and many times it’s hard to say whether one is.

For this text, though, let’s assume we aim for democratic self-governance. The hard question we aspire to answer is how we start with One founder and end up with many collective owners while also being a sustainably self-managed group of people.

\subsection{I. Step 1: Get people hyped}

The cause drives the community. The One is firstly part of the cause. This cause will probably already have some established gathering places (virtual or IRL) in some form or another.

Even though our economy is competitive, our communities can only survive through cooperation. Interdependence is crucial and through that, network communities can flourish. Being as independent as possible is considered important for survival, yet I have found out that it’s usually the opposite that happens. However much independence a community has, it’s impossible to survive; it is through dependencies on other communities (through the edges of a graph) that make the communities themselves (the nodes of the graph) more powerful.

This is why I think it’s an important first step for the first One to join these communities and their gatherings. This can mean a number of things in reality. Maybe the cause organises on Facebook Groups or through meetups. Or maybe they are under another umbrella cause or organisation. Whatever the case, this is where One starts.

To illustrate further, maybe:

\begin{itemize}
    \item{The cause is about fixed gear bicycles, and the gathering square for many is at a specific bike shop.}
    \item{The cause is about helping refugees in a city and people coordinate through Facebook Groups.}
    \item{The cause is about functional programming and there is a meetup group that organises presentations every month.}
\end{itemize}

Once One has joined the existing communities, they can connect with them and share their vision. This is what gets people hyped. On our examples above:

\begin{itemize}
    \item{“Wouldn’t it be awesome if we had a fixed gear bicycle race in the streets of our city?”}
    \item{“Wouldn’t it be amazing if we could actually provide shelter to the newly homeless refugees?”}
    \item{“Wouldn’t it be great if we had more people write in functional languages and subsequently raise the quality of our libraries?”}
\end{itemize}

Once people are hyped, they are ready to dedicate themselves to the cause. They are ready to spend a few hours per week or per month to realise this awesome vision that was conceived and shared with them.

\subsection{II. Step 2: Share ownership}

Once there are some additional members, the first thing the One will notice is that this is the One’s thing. It’s their baby. They made it an entity and the fellow associates try to improve it. Some may contribute a bit, others more, but maybe no one as much as the first One. This is a crucial point. It’s when the One might want to share ownership. In this way, the One actively shows to the associates, who believed in them and followed them, that this is their baby too. The babyness metamorphoses into a distributed essence.

In order for the One to share ownership, they need to actually share the ownership. Hand the keys to the associates, whether those are login credentials or actual keys to an office; or maybe adding them as directors to a limited company. Convince them with actions that it’s their baby too.

In my experience from the past decade, people embrace that. They respect it and do not betray it.

To judge whether the ownership sharing process was successful is to ask whether the concept of the One within the community has died.

Once that’s done, the community will be entering — among other things — a spiral of informal and formal ceremonies. The formal ones are usually the essence of the community. For example:

\begin{itemize}
    \item{Cycling communities go for bike rides. Maybe every Friday at 9 pm. That's the formal ceremony. But some evenings, people gather at the aforementioned bike shop and talk and drink beers. This is where they schedule the bike rides. This is the informal ceremony.}
    \item{Refugee Facebook groups donate food and clothes and help refugees. They have a weekly meeting where they gather all the food and clothes and deliver them to the shelter (formal ceremony). Every day at lunch and afternoons, they learn about the news of the cause, think of new ways to help and share views (informal ceremonies).}
    \item{Programming communities do meetups or hackathons or conferences. They have scheduled events (formal ceremonies) or casual hangouts offline or online (informal ceremonies).}
\end{itemize}

It’s a spiral because it’s a cycle yet also slightly different every time. There is progress to it. There is a certain directionality, which is hopefully towards improvement.

Once these ceremonies are set, they are hard to change. They become part of the culture of the community. This is important to be aware of in case of the ceremonies not being nice.

Even though ceremonies are processes, they are not explicit processes. They are implicit processes and this is the reason that they are not enough to fight inherent entropy.

\subsection{III. Step 3: Establish processes}

To tackle inherent entropy, a community needs processes—the explicit kind.

These are the boring bits which, if not existent, the harder it will be to keep the community from withering away. It will also not be obvious the community needs processes until too late. Thus, they need to be set early. Too early and people will lose their enthusiasm from the bureaucracy; too late and they won’t be bothered to participate at all.

These processes establish the community, like a ship sailing on its own. They need to include answers to things like:

\begin{itemize}
    \item{How do we get new members?}
    \item{How do we introduce new projects?}
    \item{How do we make decisions collectively?}
    \item{What kind is our entity towards others, such as the city, other communities, the government?}
    \item{What are some of the values we share and actualise in this community?}
    \item{How do we handle disagreements?}
\end{itemize}

Furthermore, it should be noted that in this balance of processes that needs to be achieved, there is another factor that usually causes derailment. Arguing for the point of arguing and not for the point of practical progress. Some people are more susceptible than others in this, but in general, there needs to be active re-evaluation of the scope of the processes.

\begin{quote}
    What if someone evil comes and decides to do a hostile takeover of our community?
\end{quote}

The above is an example of a question that tends to extend the scope of process definition infinitely. My advice is to contain these kinds of discussions, either temporally (e.g. timebox them) or by their cardinality (designate a working group of a few people to figure it out).

The general rule is to create processes for the current people and the current problems. Beyond that, it can wait.

Now, to provide an opinion on the question above, I think the balance is between shielding old members and sharing ownership with the new. If you shield too much, then you’re less democratic and new members are not as genuine members. If you share ownership too much, then it’s easier for new members to change the community faster than the community’s culture has got to them. I generally vote sharing ownership more and risking takeovers as—usually—moving into a new community is easy enough.

\subsection{IV. Collective ownership}

Democracy is hard but I think it’s worth it. Especially for communities which usually begin with volunteer work, democracy is both of vital importance and — sometimes surprisingly — the default form it starts with. In addition to that, the fewer people there are, the easier democracy is to implement. For instance, in a city or country level, implementing democracy is orders of magnitude harder. Thus, if one values democracy, it’s worth practising it in a small community.

The crucial thing to accept in these kinds of communities is that this is everyone’s baby and that ownership only exists as shared ownership. Even though one started it, or one has made the most significant contribution, or one fights for it more, it’s still fair and possible for that person to be voted out tomorrow. Avoiding that is definitely an objective, but accepting it is the key. This acceptance is what solidifies the distributed ownership.

\begin{quote}
    Freedom is not worth having if it does not include the freedom to make mistakes.

    — \emph{Mahatma Gandhi}
\end{quote}

\section{Open source as societal theory}

From the website of the Open Insulin Foundation\footnote{https://openinsulin.org/}, an organisation which originates from California, United States:

\begin{quote}
    We’re a team of biohackers with a variety of backgrounds, and skills, and relationships to insulin and diabetes from [\dots] around the world.

    We’re working to develop the first practical, small-scale, community-centered model for insulin production to make insulin accessible to all. [\dots]
\end{quote}

and:

\begin{quote}
    We envision a world [\dots] where people living with diabetes and their communities can own and govern the organizations that produce the medicine they depend on to survive.
\end{quote}

The problem is well-known to the Western world. Pharmaceutical companies are taking advantage of the US citizens. All of them through taxes. Some of them more directly through having to pay high prices for a cheap drug, insulin, which is literally vital to their livelihood.

The people of the Open Insulin Foundation want to solve this problem with the most direct and effective solution they can think of: teach everybody who needs it how to create it—full stop.

Somebody commented about them:

\begin{quote}
    Wouldn’t it be easier to lobby the Congress to fix the laws?
\end{quote}

The US Congress is probably aware of the issue. The Open Insulin Foundation is direct action. It reminds me of something else.

\subsection{I. Open source movement}

I think the open source software movement is like this.

In 2003, somebody thought it’d be a good idea to write software for a newspaper website using Python. Then, they gave it to the world, for everyone to use it and solve the same problem they faced—for free. Not only that, but there are a good amount of people who keep improving it (now, 20 years later) without immediate reward. I’m talking about the Django web framework\footnote{https://www.djangoproject.com/} here, just one among the countless open source projects which serve as the bedrock of the modern software industry.

All of these projects, as part of the open source software movement, work in a gift economy. Their creators expect nothing. They just build, maintain, improve—rarely asking for rewards.

\subsection{II. Motivation}

I can see (at least) two potential motivators. One: they are happy to help. If they have already solved a problem, why not release the code so that someone else can find the solution instead of someone having to solve it again?

Two: they believe in the open source movement. Politically paraphrasing: they believe in the power of the gift economy. When I was starting to learn coding, the open source movement wasn’t as big as today. All I wanted was to find some code to read, copy, understand how it works, and change it, for my own ideas to materialise. Turns out copying code is not legal—unless it’s an open source piece of software\footnote{This is not exactly true, either, though. There is a wide array of licenses that define more specifically what one can and cannot do with some code.}.

Open source software was the most exciting thing for me back then because it meant that I can read it, understand it, change it. This is the gift I received and this is the gift I want to give back now that it’s my turn. I think a large percentage of the open source movement shares this kind of motivation.

In addition to this simple “give back” mentality, many also believe in the movement in a more political way. Free software is similar but not exactly the same\footnote{Read more at https://www.gnu.org/philosophy/open-source-misses-the-point.en.html} as open source.

\subsection{III. Impact}

It’s important to clarify, here, the range to which the free and open source software movement extends. Anyone (however inexperienced) can release a project as open source\footnote{And/or free software.} but this doesn’t mean that there isn’t an abundance of world-class open source projects. Django, as the example already mentioned, is widely successful and used by lots of companies the average non-programmer Western citizen knows, eg. Instagram, Spotify, NASA, et al.

To reiterate more clearly: all of these companies (and practically all internet companies) use—not just as a nice supplement but—as part of their most essential and critical infrastructure software that was made in the past and released for free and for anyone to use.

For example, Instagram, as one of the most extreme examples, makes billions using Django. Yet the people who created Django have not received much from them\footnote{At least I couldn’t find any indication that they have. One can see the corporate members of the Django Software Foundation in this website: https://www.djangoproject.com/foundation/corporate-members/.}. I am not interested in presenting this as unfair. The Django authors shared their creation for others to use without any expectations—as a gift! My claim is that we should admire their eminence and maturity. It’s a true ethical achievement.

\subsection{IV. Applying it to society}

The Open Insulin Foundation applies this philosophy (of the gift economy) to the non-programming crowd and specifically to the healthcare industry. They face the problem head-on and fight fiercely to solve it with no fanfare, only essence.

\begin{enumerate}
    \item{We need insulin.}
    \item{Let’s make it for everyone.}
\end{enumerate}

There’s nothing more to it.

This might make us ponder: can we apply this thinking to more things in our societies? Can we make food for everyone and be done with this issue? Can we make houses for everyone and end the constant stress and misery?

I feel like this is a nice\footnote{Very few people would agree with the statement that this is a nice first step. Not only advocates of capitalism, but even its opponents would claim that this is not a viable alternative. To which I poetically respond by paraphrasing Devine Lu Linvega: “everything is dark under the ultraviolet sun” (https://merveilles.town/@neauoire).} first answer to the ever recurring question: if not capitalism, then what?

Maybe gift economy.

\chapter*{Interlude II:\\Atom Heart Mother}
\addcontentsline{toc}{chapter}{Interlude II: Atom Heart Mother}

\emph{Theme from an Imaginary Western}\footnote{David Gilmour had given that name to the chord progression of the main theme of \emph{Atom Heart Mother}}. Progressive symphonic perfection. Such is the transcendence, that the band\footnote{\emph{Atom Heart Mother} is a song by the English rock band \emph{Pink Floyd}.} has been now criticizing this piece of their work, many years after its release.

Inspired by a pregnant woman who had been fitted with a heart pacemaker, Atom Heart Mother provokes the subject of sapiens-made devices enabling the — still? — unexplained miracle of this world: life, soul, consciousness.

The lyrics follow.

\begin{quote}
    Fah\\
    See co ba\\
    Nee toe\\
    Ka ree lo, yea

    Sa sa sa sa sa, fss\\
    Drr bo ki\\
    Rapateeka, dodo tah\\
    Rapateeka, dodo cha

    Ko sa fa mee ya\\
    Na pa jee te fa\\
    Na pa ru be, mm\\
    Ba sa coo, ba sa coo

    Ba sa coo, ba sa coo\\
    Oo\\
    Ku-ku loo, ku-ku loo\\
    Yea yea yea um

    Hm ku-ku loo you\\
    Too boo coo doo\\
    Foo goo hoo joo\\
    Loo moo poo roo

    Here is a loud announcement

    Silence in the studio!
\end{quote}

Apart from the two lyrics in English, all other are spoken out of authentic exclamation for the duality of our nature in the cosmos; thus the absence of any specific language, and the presence of the universe-al one.

From animals and early humans stems the intrinsic notion of survival; eat, live. However, starting with societies of just 3 millennia ago, we concern our minds with questions such as humanity’s purpose or the will of the gods. Our lives are drawn between the edges of “wake up, eat, sleep” and “commit suicide because you have no purpose”. These are the reflections of the aforementioned duality, and \emph{Atom Heart Mother}, an ode to this absurdity that perpetuates our lives ad infinitum, achieves something nobody else has achieved: the musical manifestation of humanity’s desired cosmic balance.

Fuckin’ A.

\chapter{Rethinking the Industry}

\section{The strong state collapse}

Bitcoin is a digital currency, which many people are very excited about. Why is it exciting and is it, really?

The technology behind Bitcoin and especially other cryptocurrencies is seriously impressive.

The—now—legendary Satoshi Nakamoto figured out how we can build digital currencies without trusted third parties in 2007. This was considered a fundamental challenge for a while. Nobody could figure out a design in which no one needs to trust anyone while still having transactions and money online. But Satoshi did it! What a legend.

\subsection{I. Bitcoin is a store of value}

Many proponents of Bitcoin claim that it is a store of value. When I first read that I thought “well, everything is a store of value; EUR, USD, my car, all of them, in a way, store value”.

That’s not what they mean, though. They mean store as in \emph{permanent store}. What’s the point of money if one is not spending it but permanently storing it? Good question.

There is a problem that comes up if one has more than a few months’ worth of money. Their location. In the bank, someone with less money would say. In the pocket as cash, someone with even less money would say.

That, with which both poor and rich people agree, is: fuck banks.

Banks can decide that they don’t allow you access to your funds just because. Governments can make banks do that. What if we had a way to store money and 100\% control it?

Gold is valuable, tradable, and can be 100\% owned—I can just store it in my basement. If the banks don’t like me (or if they collapse, or if someone attempts a coup, or if civilisation collapses), one thing is for sure: I \emph{own} my money and I’m still free and independent to buy stuff.

Now we’re getting somewhere. There is only one problem with gold. It’s heavy to move around. What if we had gold, but digital? Enter Bitcoin.

Bitcoin is virtual gold; it’s supposed to be hard to move around—in a way. For instance, Bitcoin has high transfer fees\footnote{The cost of Bitcoin’s transfer fees has historically been very volatile. It depends on if a lot of people want to transfer money, which makes the Bitcoin network crowded (and thus slow). It also depends on whether the sender wants to verify a transaction faster (higher fees mean higher speed as miners choose to verify high-value transactions first). In 2022, the average Bitcoin transfer cost was between \$1 and \$2, however, it has reached prices higher than \$50 at certain periods across the years.}, so if you want to send Bitcoin (not matter how much!) to someone, you have to pay an extra amount for the transaction itself.

The universal hate for banks is partly because they make consumers pay fees, but unfortunately Bitcoin has the same problem.

However: Bitcoin is for value storage, not transfer. Maybe billionaires don’t care, they can hide around the whole globe. But what about a millionaire, or a thousandaire? They could hide, but they may also be hunted by tax offices. Maybe one day said tax offices will find their hidden funds and eradicate them. It would be nice to actually control them.

\subsection{II. Bitcoin will not help people who need help}

In addition to banks, the rich and the poor might actually agree to “fuck governments” too.

Let’s imagine a future potential Bitcoin-enabled socioeconomic landscape. The price of Bitcoin is now \$2.6 million. Governments ask tax residents to declare all their wallet addresses. Bitcoin mixers are not legal. Power may now mean having access to the illegal ones. There are now two outcomes. Either we give up on taxes or we find a way to at least monitor (if not control) the Bitcoin network.

In the first case, the concept of the strong state dies. At least I can’t see how it couldn’t. No taxes means no public schools, no free universal healthcare, no furlough when the next pandemic comes, and generally no control of the economy whatsoever. Some claim this is when we’ll finally be free.

In the second case, Bitcoin is just another currency, just like USD and EUR. Central banks control it, ministries decide what to do with it, and the market can be as free as the government allows. Essentially, the same thing we have now.

Is there a third case? Maybe people decide they want strong states and declare all their income themselves. Or maybe BTC remains a non-mainstream thing. Or maybe something more radical happens, which I cannot predict.

I see mostly the following potential outcomes. Those who are super rich remain super rich. Those who are generally well off may become super rich. Those who are poor or really poor will remain as such—if not worse.

\subsection{III. Bitcoin does not help people who need help now}

To make the present case of Bitcoin worse:

\begin{itemize}
    \item{One cannot pay either online or the supermarket with Bitcoin}
    \item{It is much slower than the current financial systems}
    \item{It is much less secure and does not forgive mistakes}
    \item{It has very high fees for consumer-scoped transactions}
    \item{It significantly contributes to global warming}
\end{itemize}

We have to give up all the above in order to benefit from 100\% controlling our money—which we cannot spend directly but only store it.

Bitcoin becoming more expensive over time may be a reason to buy. “Investing” in the idea that more people will be convinced to be bank-independent. However, any short term (or even long term) gains not backed by value in essence (either social or even purely material) are not interesting to me. Until we can figure out how Bitcoin can \emph{actually} contribute to world society, I denounce it.

\section{Industry and context}

\subsection{I. Technology}

I love technology as essence, as the way of making things that on the surface seem magical. I hate it as an industry, of what “big tech” has come to mean the last decade and how technology companies affect the world.

I considered these two as separate concepts in my mind, yet as time passes, they have converged. How can one love technology if the only way they can interact with it is through companies that not only do not care for societal betterment but also actively contribute to its detriment?

Google invades our privacy to sell ads. Amazon exploits warehouse workers for next day deliveries. Facebook sells ads to anyone—however malicious—that buys them. The examples are many.

The majority of my friends in the domain agree. They are sad about their work and our industry. Many are worried that maybe they are actively making the world worse, after all. Most of the time, it’s hard to know; things can spiral out of control even if in the beginning everything seems to have the right moral compass.

\subsection{II. Context}

But I think the problem is not the industry nor technology as essence. It’s the context it exists within. And this context is the same for all industries: our societal and economic system.

\hyphenation{nur-ses}
Ask anyone; how do they feel about their industry? The responses I got were almost always negative. Technology? Silicon Valley thinks it’s saving the world. Healthcare? Nurses (and/or doctors) being paid meagre wages while doing an extremely hard and valuable job. Art? Millions of artists unappreciated and unable to live off their work and a small elite controlling the industry. Hospitality? Low wages for tiring and intellectually non-stimulating work. Transportation? Delivery? E-commerce? Energy? I think the reader can fill in the answers from their own experience.

All of these problems have one holy mantra as a common denominator: \emph{we have to maximise our profits}.

Everything else trickles down from there. Google and Amazon—and every other company—have to increase their profits quarter to quarter, however ludicrous they already are. They have to find that 0.001\% optimisation that will save them millions. This could mean shortening a worker’s 23-minute break to 21 minutes, or tracking users in this new, slightly more creepy way.

\subsection{III. Beginnings}

These companies did not necessarily start evil. They began fresh as the underdogs on a mission to make the world a better place. Google’s mission was \emph{to organize the world’s information and make it universally accessible and useful}. Their contribution was definitely beneficial in many ways.

Amazon’s mission was to \emph{offer customers the lowest possible prices, the best available selection, and the utmost convenience}. Good outcome for customers. Ordering anything and having it tomorrow on my doorstep is certainly utopian for the customer. But what about the workers?

This doesn’t mean—of course—that this is an argument against progress. I’m only against that which exploits others to achieve its end. In this case, we exploit many to have next-day delivery and we even exploit ourselves\footnote{We exploit ourselves by allowing ourselves to be spied upon while relinquishing control of our own data.} to have a web search engine and free email. We can progress without exploiting others and we should do that even if it takes more time and effort to do so.

All this, in the end, seems like an instance of instrumental convergence. Nick Bostrom’s paperclip maximiser is the thought experiment that comes to mind:

\begin{quote}
    Suppose we have an AI whose only goal is to make as many paper clips as possible. The AI will realize quickly that it would be much better if there were no humans because humans might decide to switch it off. Because if humans do so, there would be fewer paper clips. Also, human bodies contain a lot of atoms that could be made into paper clips. The future that the AI would be trying to gear towards would be one in which there were a lot of paper clips but no humans.

    — \emph{Nick Bostrom, 2003 (paraphrased)}
\end{quote}

For our socio-economic system, the paperclip is the GDP or a company’s quarterly revenue. Everything at the altar of maximising profits—whatever it is, it’s less important than profit, anyway. It doesn’t have to be this way, though. We can all agree to designate human wellbeing as the paperclip, rather than GDP or quarterly revenue. Hopefully, such a policy change seems more easily achievable than next-day delivery of all items humanity produces.

\section{On startup acquisitions}

One of my favourite companies got acquired recently\footnote{I'm referring to ActiveCampaign’s acquisition of Postmark in 2023.}. I’m always trying to figure out if I want to be happy when a company gets acquired.

It’s usually acquired by a much bigger company. Founders and investors make a lot of money. Everyone looks ecstatic and everyone congratulates everyone else. But why?

Why should we be happy they got acquired? Presumably, because of the money they made.

Silicon Valley’s imaginary is based on the fact that we improve the world with software and technology. Money is irrelevant. What matters is to advance as society.

It’s been a while since I started viewing the process of building a company as providing a service. Not as in system service, but as in community service. Someone who creates a company is someone going out of their way to help people. In return for the product/service provided, these people pay money. Both to keep providing it and as a thank you.

So, when this model ends, I think it’s a sad day. People who really cared about something and dedicated themselves to it, stopped. Maybe they got tired. It’s ok to get tired. But probably not a cause for celebration.

\section{A new startup lifecycle}

\subsection{I. Posthaven’s philosophy}

Posthaven\footnote{https://posthaven.com/} is pretty awesome. From their pledge\footnote{https://posthaven.com/pledge} section:

\begin{quote}
    We’ll never get acquired. [\dots]\\
    We’ll never raise money. [\dots]\\
    Posthaven is a long-term project that aims to create the world’s simplest, most usable, most long-lasting blogging platform.
\end{quote}

Posthaven has been around for a while and I would always appreciate its simplicity when coming across a blog built with it—it’s not a very popular blogging platform, so that wouldn’t be too often.

In the quote above, not getting acquired and not raising money are presented as arguments for inspiring confidence. Confidence, in that this project can indeed last for a long time and remain sustainable.

Posthaven hadn’t changed its landing page for many years — until a few months ago, that is. When I would visit their website, I would question whether they are still committed to the project, or they have given up. This latest renewal of their online presence made me stop worrying, though. They certainly haven’t given up.

I never knew who was behind it but this time I read its story. Two SV founders had made a blogging startup\footnote{Their blogging startup was called Posterous.} which was sold to Twitter in 2012. Twitter shut it down after acquiring it and this made them sad. They then decided to do it again, only this time: no VC money, no big teams, no acquisitions; just a product with paying customers, not destined to capture the market or make a lot of money.

If they don’t have to own the universe, they can just be a red ocean\footnote{In the book \emph{Blue Ocean Strategy}, authors W. Chan Kim and Renée Mauborgne talk about red oceans, i.e. all industries in existence today, the known market space in contrast to blue oceans, the industries and markets which are yet to be discovered. Startups usually aim to explore blue oceans by creating new markets (and needs).} shop which achieves long-term sustainability with significantly less effort.

\subsection{II. Reasons for acquisitions}

I always hate it when good companies get sold. GitHub, Bandcamp, Keybase, Postmark, Figma, SwiftKey, even Zenly; the list is endless. The story needs no repetition but I will: they either get shut down and/or absorbed.

Everybody celebrates acquisitions. The founders and early employees make a ton of money and the acquirer becomes bigger and better and more profitable. In other words, more centralisation.

What’s the point of alternative non-big-tech products and companies when eventually their success is a big-tech company acquiring them? Is the point founders becoming rich? Is the point independent innovation because big-tech corporations are slow to prototype and/or iterate?

Maybe there is no specific point. Only that people get tired of running companies. Selling is a way out. One that will make you rich and your future life more comfortable. One that will also validate your work in the industry and in society. You were bought; you were definitely doing \emph{something important}. This is not only about money. It's also about reputation; and impact; and ego.

\subsection{III. Posthaven’s founders}

Say, we build an internet product and we keep it online for 10, 15, even 20 years. But eventually we get tired. Even if not, we die. That’s when all bets are off. Our internet product either gets sold or just dies.

Maintaining an internet product is hard. People who do it need to be pretty well motivated. The people who made it usually are. The people who use it might also be.

Just like the Posthaven founders, I have also made a blogging platform, mataroa.blog\footnote{https://mataroa.blog/}. Just like them, I also do not want to sell it or make tons of money off it. Its whole point is to provide a quiet place for people who want to explore saying something on this incredibly loud internet where nothing is heard. Everybody shouts; let it be us who try to whisper instead.

\hyphenation{ma-ta-ro-a}
But maybe after 20 years I get tired. Maybe at some point I don’t want to have to think about maintaining mataroa any more. What do I do? Some people really like mataroa and, in a way, depend on it. I’d rather not let them down. Everybody knows the feeling of companies shutting down people’s favourite products. I don’t want to be someone who causes this.

Selling would imply finding someone who believes in the philosophy behind it. But I seriously doubt people interested in buying internet products are people who care about a philosophy of a product. They most likely buy because they believe they can increase its profit and/or possibly flip it.

I spent many months building mataroa. Its current annual revenue is \$540\footnote{https://mataroa.blog/modus/transparency/}, which means I would need multiple decades to break even. It also means that it would be extremely cheap, if on sale. Had it been 10x or 20x more successful, though, we might have had some interesting purchasing scenarios to discuss.

Whatever the case, the question we want to answer remains: what is an alternative solution to businesses outgrowing their founders?

\subsection{IV. Alternative futures}

If and when I become tired of running mataroa, I’m thinking of executing the following plan.

\emph{1.} Gather interest from people wanting to run and maintain the mataroa platform. People interested write a single text that explains their motivation and capability. All texts get published.

\emph{2a.} If nobody is interested, that’s the end. Users are given a year or two to migrate and we shut down everything after.

\emph{2b.} If people are interested, we, all together (myself, the mataroa users, and the people interested), discuss and hopefully reach consensus as to who is the best person (or people)—among those interested—to become the future owners and maintainer(s) of mataroa.

If we can’t reach consensus, we vote. If there is a clear winner, so be it. If there isn’t, we fork. 2 or 3 new versions of mataroa appear and users choose to migrate to their preferred server.

The most interesting part of this plan though is that the new maintainers and owners of mataroa will not buy mataroa from me. I will gift it to them. The code is open source, but the gift also includes the existing server instance, domain name, and all users and their paying subscriptions.

The goal here is to shift the current social imaginary from one kind of transaction to another.

The transaction I want to move away from is: we exchange ownership and control of a product and platform with a lump sum of money.

The transaction I want to arrive at is: one promises to treat something with respect to its philosophy and they receive in return control and ownership of this product and platform.

\subsection{V. Founders outlived}

Maybe you think that’s mad, and I don’t blame you. But I hope to convince that it’s much less mad than it might initially seem.

Posthaven, the blogging platform which goes against all SV ideals, was not founded by SV outsiders. One of its founders is Garry Tan, the current CEO and president of Y Combinator, a company that has been instrumental in developing the SV model of entrepreneurship. It is him and Brett Gibson—another VC—who chose to create Posthaven with the pledge of no VC money and no acquisitions.

They choose to run a business that is so boring and non-VC-oriented when their whole life is—at present—about the VC model, one of high growth, market capture, and exits.

We could say they are—in addition to the VC model—interested in something else. An alternative way of startup life. One which maybe promotes values that are in conflict with the VC model.

But how does this alternative way work? Can we find out how to have internet startups that \emph{outlive their founders}, while at the same time they do not turn into profit maximising machines?

This is a need for space for otherwise. I think that’s quite interesting and I think it’s a glimpse into the future—rather than the past. A glimpse into a world beyond capital.

\chapter*{Interlude III:\\Everything Everywhere}
\addcontentsline{toc}{chapter}{Interlude III: Everything Everywhere}

\emph{Everything Everywhere All at Once}\footnote{\emph{Everything Everywhere All at Once} is a film written and directed by Daniel Kwan and Daniel Scheinert, released in 2022.} is a brilliant depiction of post-modern life in a post-modern way.

We have all the answers to our problems; we know we do. They are what the movie says. Love. Ignoring everything except one thing. Tackling depression (depicted as the bagel). Resisting fighting because we’re confused.

We live everything, through the non-stop internet. We live everywhere, through teleportation, also known as flying. It’s all at once—it is. This movie is our life, just through another lens. One through multiverses. Why? Because we can only understand through technology; through hard facts, and our current theory of everything is string theory, multiverses, quantum superposition, et cetera—so that’s the only way we can be convinced of love being the point of life.

In this 2020s version of Matrix the protagonists are female, a mother and a daughter; a present-day background, an antithesis to the world’s patriarchy.

\begin{quote}
    Then I will cherish these few specks of time.

    – \emph{Evelyn to her daughter, Joy, just as she’s suiciding}
\end{quote}

Not only is the above line a depiction of depression (everything is black; sometimes, yes, we might have fun. But everything really is black) but also a depiction of eros. Not love, not this umbrella word describing a generic feeling. But eros, \emph{being in love}, which is total agony (i.e. blackness) while raving for the moments when you’re together with your erotical counterpart. Just like stars. Specks of light amidst the black blackness.

The husband’s key fighting technique, kindness, is only the beginning of the movie’s thesis. Like another Jesus (and Socrates) he says “be kind! Especially when we don’t know what the fuck is going on—which is always the case, even when we’re pretending we know who we are”. Evelyn, the mom, is convinced—she’s been in love in the past and can remember not only feeling different in that present but also feeling amazing for that future. She knows this can happen.

Joy, the daughter, who in reality is the lack of joy, hasn’t. She’s the average western teenager, especially from the US, where all other teenagers are happy and rich while you’re not. There never was a good outlook for her life. She only knows her mother’s intense negativity as a response to what’s life throwing at them—and her father’s naive kind-heartedness. That’s her thesis she wants to be an antithesis to—and she does that by pushing her limits in the Alpha universe and by getting away from her parents on the IRS-enabled one.

But she fails. She fails everywhere, and at everything—that’s the nature of the true blackness of depression. She’s not convinced by her father’s proposition (which is “I choose to see the good side—please choose too”) at all. So now, the omnipotent mother will \emph{show her} her love everywhere, with everything, and all at once. Even when she’s actually the IRS agent as her husband in the hot dog finger universe where they ejaculate cheese.

That’s where the pinkfloydian wall finally breaks.

\begin{quote}
    So, what? You’re just gonna ignore everything else? You could be anything, anywhere.

    – \emph{Joy to her mother, Evelyn}
\end{quote}

But it’s negativity that’s giving us the frame whereas now everything is positive (“you could be anything! anywhere!”).

\begin{quote}
    Why would I want to be anything, anywhere, if there is no lack of it, needed to paint the outline (and meaning) of my doing?

    – \emph{Evelyn, in the parallel universe that she has read Hegel}
\end{quote}

Just as she reconciles the paradox of life, she says:

\begin{quote}
    We can do whatever we want. Nothing matters.

    – \emph{Evelyn}
\end{quote}

Optimistic nihilism is, indeed, many times a great point—yet can we resist the temptation when we don’t have the clarity of the superposition of depression’s thesis and antithesis? I’m not sure.

Smart people of the past weren’t sure either—that’s why they had \emph{trust} that it is like that. They had \emph{faith} in a God that told them that there is a point to all this, especially when life is not fun. We have no God so…

So, I wonder, what happens when the IRS agent makes Evelyn’s life too hard? It’s not going to be a walk in the park from now on, right? She still has to live her life with all the sufferings. But now she has the power to jump into another universe, so why not jump? If this universe is too hard? Maybe it’s ok?

We all know what happens then—she gets lost in the multiverse chaos. Maybe she wouldn’t because she has gained universal enlightenment and can fight the urge off (although we do see her mind wander right at the very end) but what about me, dear reader, when that movie gets slowly forgotten — how do I remain optimistic with nihilism?

Should I just watch this movie again? What if I create a symbol to represent this movie in my mind and just think of that? Say, this symbol is the word “hotdog”. I wouldn’t need to go through the process of seeing and feeling this movie again, I could be reminded of it with just “hotdog”!

But what if instead of “hotdog” this word is “god”?

\chapter{On Freedom}

\clearpage

When I was at school, I felt content with life. When I started introspecting my personal relationships, I couldn’t understand how they could have been limiting (or liberating, for that matter). When I started getting paid for work, I thought, \emph{“this is it — I made it”}.

In all those cases, I was expected to not complain, and I didn’t. It was over a longer period of time that I discovered how limited I was across all these areas without realising it.

This realisation holds a certain essence of freedom.

In another scenario, I return home from work and turn on the TV. There is a pizza ad that makes me realise I want to eat pizza. I order it and eat. Would I have eaten pizza if not for the ad?

There is a difference between eating pizza and not realising why and eating pizza while knowing it’s because of the ad. This difference is a difference in freedom.

Understanding the reason for something being the way it is matters in the context of freedom. A first step in becoming free is the realisation that one is constrained. As society members, we may be constrained by school, work, our relationships with friends and with family members. Once we realise what constrains us, we can act. If we never realise, we can never be free. We remain controlled by that which affects us without our understanding.

In the TV pizza example, there wasn’t someone explicitly aiming to make me eat pizza that day. Nobody was explicitly controlling my life other than me. However, assuming I ate the pizza because of the TV ad, I may no longer had explicit power over my dietary preferences. Since nobody else had that power either, as we claimed in the previous sentence, this means that \emph{that power was lost}.

Here’s the harsh reality, though: even if we gave up that power and nobody picked it up, that power is still our responsibility. It’s harsh because such responsibility is overwhelming. Yet there isn’t anyone else to bear it. No supreme forces, no experts with profound knowledge, and no high-tech answers. It’s just us. If \emph{we} don’t act, nobody will.
